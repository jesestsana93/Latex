% leaflet se usa para triptico
%5-6-1-2-3-4 posiciones para el triptico
\documentclass[10pt, notumble, letterpaper]{leaflet}

\usepackage[utf8]{inputenc}
\usepackage{graphicx}
%Paquete para darle color al documento(secciones)
\usepackage[usernames]{xcolor}

\pagestyle{empty}

\title{Congreso Nacional de \LaTeX}

\author{Carlos Pérez}

\date{10-12, febrero, 2015}

%Marcas que dividen al triptico
\CutLine*{1}% linea punteada sin tijeras
\CutLine*{3}% linea punteada sin tijeras
\CutLine*{4}% linea punteada sin tijeras
\CutLine{6}% linea punteada con tijeras

\begin{document}
\pagecolor{yellow}

\maketitle

\thispagestyle{empty}

\section{Talleres}

Talleres disponibles:

\begin{table}[h!]
\begin{tabular}{ccc}
 Taller & Costo & Fechas \\
 \LaTeX{} & \$ 500.00 & febrero 10 \\
 Excel & \$ 500.00 & febrero 11 \\
 Linux & \$ 500.00 & febrero 12
\end{tabular}
\end{table}

\newpage
\pagecolor{white}%Aqui cambia nuevamente de color

\section{Sección 2}

Texto en la sección 2
\begin{center}
\includegraphics[scale=0.2]{escudoCiencias}
\end{center}


\newpage

\section{Sección 3}

Texto en la sección 3

\newpage

\section{Sección 4}

Texto en la sección 4

\newpage

\section{Sección 5}

Texto en la sección 5

\newpage
\pagecolor{blue}
\section{Sección 6}

Texto en la sección 6

%Son 6 secciones en el triptico
\section{Sección 7}

Texto en la sección 7

\end{document}
