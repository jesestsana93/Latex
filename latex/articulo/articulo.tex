\documentclass[12pt]{report} % Clase de documento: artículo y tamaño de letra
 
\usepackage[spanish]{babel} % Manejo de idiomas
 
\usepackage[latin1]{inputenc} % Escritura en castellano con acentos
 
\title{Mi primer art\'iculo} % Título
 
\author{Quien escribe} % Autor. Pueden ser varios agregando \and Otro autor
 
\date{\today} % Fecha siempre actualizada al día presente al compilar.
 
\begin{document} % Inicio del documento
 
\maketitle % Hace la portada
 
\tableofcontents % Hace el índice de contenidos.
 
\section{Introducci\'on} % Primera sección, se incluye en el índice.
 
Aqu\'i escribo la introducci\'on. Cada p\'arrafo se separa con una l\'inea en blanco.
 
\section{Cuerpo del art\'iculo} % Otra sección.
 
Puedo hacer que el texto vaya en cursiva con \emph{texto en cursiva}. Hacer una enumeraci\'on:
 
\begin{enumerate}
 
\item Linux
 
\item Unix
 
\item FreweBSD
 
\end{enumerate}
 
Las notas a  pie se hacen con\footnote{Texto que aparecer\'a en la nota a pie de p\'agina.}.
 
\subsection{Comandos}
Los comandos como part, chapter, section, subsection, etc, sirven 
para dar estructura al texto. 

\subsection{P\'arrafos}
Por cierto, los p\'arrafos en \LaTeX{} se separan mediante l\'ineas 
en blanco, no importa cuantas. De la misma manera, no importa el 
n\'umero de caracteres en blanco que separan las palabras, pues 
\LaTeX{} s\'olo los usa para distinguir entre una palabra y la
siguiente.

 
\section{Conclusi\'on}
Aqu\'i escribo la conclusi\'on.
\begin{thebibliography}{10}
\bibitem[LATEX]{citalatex}E\'ito.Brun.Ricardo
\textit{Lenguajes de marcas}TREA.2008
%\url{http://www.matematicas.unam.mx}
\end{thebibliography}
 
\end{document} % Fin del documento.
 
Esto que escribo no ser\'a incluido en el texto porque \end{document} le ha dicho que no contin\'ue con lo que sigue.
