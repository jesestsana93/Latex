\documentclass[12pt]{book} % Clase de documento: libro y tamaño de letra
 
\usepackage[spanish]{babel} % Manejo de idiomas
 
\usepackage[latin1]{inputenc} % Escritura en castellano con acentos

\title{Mi primer libro}
 
\author{autor del libro, quien escribe}
 
\date{\today}% Fecha actual
 
\begin{document}
 
\maketitle
 
\frontmatter
 
\tableofcontents
 
\chapter{Introducci\'on}
 
Aqu\'i escribo la introducci\'on. Cada p\'arrafo se separa con una l\'inea en blanco.
 
\mainmatter
 
\chapter{Primer cap\'itulo}
 \section{Listas enumeradas}
 Una lista enumerada se escribe dentro del ambiente enumerate, los 
 n\'umeros son calculados autom\'aticamente:
\begin{enumerate}
 \item El nombre de la rosa
 \item El p\'endulo de Foucault
 \item La isla del d\'ia de antes
\end{enumerate}

\section{Listas no numeradas}
El ambiente itemize se\~nala cada elemento con una ``bolita"
\begin{itemize}
\item Minix
\item Linux
\item AIX
\end{itemize}

\section{Listas descriptivas}
En una lista descriptiva, el comando item toma como opci\'on un 
texto que aparecera resaltado antes de cada elemento de la lista:
\begin{description}
\item{H} Hidr\'ogeno
\item{He} Helio
\item{Li} Litio
\end{description}

\section{Conclusi\'on}
Para concluir diremos que  cada uno de los ambientes de lista se 
puede anidar consigo mismo y con los otros ambientes de lista.


 
\chapter{Segundo cap\'itulo}
\section{Griego}
La mayor\'ia de los comandos para escribir un s\'imbolo tienen 
nombres m\'as o menos mnem\'onicos. Por ejemplo, las letras del 
alfabeto griego se escriben con los comandos formados por la 
diagonal inversa y el nombre ingl\'es de la letra en cuesti\'on:
$\alpha, \beta, \gama$, etc\'etera. Para escribir una  letra griega  
may\'uscula, el comando se forma con la diagonal inversa y el 
nombre de la letra con la inicial en may\'uscula: $\Gamma, \Delta, 
\Theta$, etc\'etera.

\section{\'Indices}
%Nota: El parrafo siguiente debe leerse como
%Para escribir sub\'indice y supe\'indices se utilizan 
%los s\'imbolos _ y ^ 
%Para que \LaTeX{} imprima en este documento tales caracteres
%se necesitan los comandos \_ y \^{}

Para escribir sub\'indice y super\'indices se utilizan los s\'imbolos
\_ y \^{}.  $a_1 = a_2 = a_3$, $x^2 + y^2 = r^2$ Cuando ambos 
\'indices aparecen a la  vez, su orden no importa, as\'i $a_1^2$ se 
ver\'a igual que $a^2_1$. Cuando el \'indice contiene mas que un 
s\'olo caracter, hay que encerrarlo dentro de llaves:
$A_{ij}= A_{ji}$.

\section{Otros s\'imbolos}
Las fracciones se obtienen as\'i: $\frac{numerador}{denominador}$, 
las ra\'ices,  $\sqrt[n]{radicando}$. Los s\'imbolos de suma e 
integral se obtienen as\'i: $\sum$, $\int$. Los l\'imites de estos 
s\'imbolos se escriben como si fueran \'indices. 

\chapter{Tercer cap\'itulo}

El entorno para crear la tabla es tabular.

En el c\'odigo fuente en la primera fila, vemos que tras iniciar el entorno tabular se introduce {l c r}. 
Eso indica las columnas que tendrá
nuestra tabla y su alineación (left, center, right). Si se quieren centrar todas las columnas, sería de esta forma {c c c}.

Las filas se dividen en cada columna utilizando el s\'imbolo \& y se finaliza con \\, dando paso a la siguiente fila.

\section{Tabla normal}

\begin{tabular}{ l c r }
   1 & 2 & 3 \\
   4 & 5 & 6 \\
   7 & 8 & 9 \\
 \end{tabular}


\subsection{Tabla con l\'ineas verticales}

\begin{tabular}{| l | c | r | }
   1 & 2 & 3 \\
   4 & 5 & 6 \\
   7 & 8 & 9 \\
 \end{tabular}

\subsection{Tabla con l\'ineas verticales, superior e inferior}
\begin{tabular}{ | l | c | r | }
 \hline                 
   1 & 2 & 3 \\
   4 & 5 & 6 \\
   7 & 8 & 9 \\
 \hline  
 \end{tabular}



\backmatter
\chapter{Conclusi\'on}
 
Aqu\'i escribo la conclusi\'on.
 
\end{document}
