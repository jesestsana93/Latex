% Este es un comentario

%\comando[opciones]{argumento}

% Aquí comienza el preámbulo
\documentclass[letterpaper,12pt]{article}

% Soporte para acentos. Se puede usar latin1 o utf8, se recomienda utf8.
\usepackage[utf8]{inputenc}

%Otra manera de colocar acentos: b\'asica
%Para la i: \'{\i}

% Español, división de sílabas
\usepackage[spanish,mexico]{babel}

% Usar codificación T1
\usepackage[T1]{fontenc}

%Este comando permite convertir cualquier referencia en una liga
\usepackage{hyperref}


% Aquí termina el preámbulo e inmediatamente abajo inicia el cuerpo del documento
%
% Ninguna de las expresiones anteriores debe aparecer dentro del cuerpo del documento
%
\title{Mi primer documento de tipo artículo en la latex}
\author{Jesús Sánchez}
\date{9 de junio de 2015}

%Cuerpo del documento
\begin{document}

\maketitle

%Indice general
\tableofcontents

%Salto de página: \pagebreak o \newpage
\pagebreak

%vspace{1cm} separación en vertical
Ejemplo de documento LaTeX de la clase {\ttfamily article} con una estructura básica. 
Incluye una sección y una subsección. También se incluyen algunos otros  
detalles básicos.

\section{Primera sección}

LaTeX nos permite crear distintas clases de documentos, entre las que se pueden mencionar: 
\textit{book}, \textit{article}, \textit{report}, \textit{letter} y \textit{slides}, entre otras.
%También puede ser \texit{book,article,report,letter y slides}

\subsection{Clases de documentos}
Todos ellos tienen diferentes características y formato. Por ejemplo, el estilo \textit{"book"} 
(libro) coloca un estilo diferente a las páginas pares e impares, permite crear documentos 
divididos en capítulos, introducir diferente  tipo de tablas de contenido, entre otros detalles.



\subsubsection{Espacios}
%\newline o \\ es equivalente al salto de linea
En LaTeX los espacios y saltos de linea no siempre se toman de forma literal.
Por ejemplo, un espacio        es      equivalente a      dos o más.\\
Un salto de linea es equivalente a un espacio en blanco.



Dos o más saltos de linea son equivalentes a un salto de linea.

Se pueden introducir saltos de linea colocando una doble diagonal invertida.

Se puede introducir un salto de pagina mediante el comando $\backslash$newpage o \textbackslash pagebreak.


\newpage
% Segunda sección
\section{Caracteres especiales}
\LaTeX %También se escribe como \LaTeX{}
%Comillas: ''
LaTeX hace uso de varios ''caracteres'' para indicar las operaciones que se llevarán a cabo sobre el texto, 
a estos se les conoce como \emph{caracteres reservados}, todos ellos tienen un significado especial 
para LaTeX, por lo que son interpretados por el compilador.

Entre ellos se encuentran:

\textbackslash

\{

\}

\$

\&

\_

\^{}

\#

\~{}

\%

A con acento circunflejo: \^a 

U con dieresis:\"u

Para hacer uso de estos caracteres es necesario \textit{escaparlos} o utilizar una expresion que los represente. Por ejemplo:

$A = \{a, b, c\}$

b\textbackslash a $= \frac{a}{b}$

\section{Acentos}

Los acentos en LaTeX se pueden escribir de la siguiente forma:
%\verb@\ lo que tiene que desplegar tal cual @
\verb@\'a@ = \'a

\verb@\'A@ = \'A

En el caso de la letra \~n, se puede introducir de esta forma:

\verb@\~n@ = \~n

\verb@\~N@ = \~N

%\noident: para indicar que no debe de aplicar sangria al parrafo
\noindent Otra forma de introducirlos es haciendo uso del paquete \textit{inputenc} el cual 
permite escribir caracteres en el formato habitual, convirtiéndose internamente el texto 
introducido a texto de LaTeX, de acuerdo con las diferentes tablas de equivalencia para las 
distintas páginas de códigos y de forma completamente transparente al usuario.

\noindent Para hacer uso de acentos, debemos poner al principio de nuestro documento, 
en el preámbulo, lo siguiente:

\verb@\usepackage[utf8]{inputenc}@

Esto nos permite hacer uso de codificacion \textbf{utf-8}, la cual es la mas recomendada por ser un estandar que esta siendo adoptado en diversas areas.

\end{document}
