%beamer permite crear presentaciones(diapositivas)
\documentclass{beamer}


\usepackage[utf8]{inputenc}
\usepackage[spanish, mexico]{babel}
\usepackage{amsmath, latexsym, amsfonts}

\usepackage{hyperref}

%Temas
%\usetheme{Rochester}
%\usetheme{Dresden}
\usetheme{AnnArbor}
\usecolortheme{beaver}

\usecolortheme{seagull}
% Para obtener más temas y colores consultar:
% http://www.hartwork.org/beamer-theme-matrix/
% http://mike.depalatis.net/beamerthemes/
% http://deic.uab.es/~iblanes/beamer_gallery/

\title{Edición de texto con \LaTeX}
\author{Juan Carlos Pérez}

\begin{document}

\begin{frame}
		\titlepage
\end{frame}

\begin{frame}
		\tableofcontents
\end{frame}

\section{Uno}

\begin{frame}
    \frametitle{¿Qué es Beamer?}\transduration{2}%Segundos a esperar en esta parte
    \begin{itemize}
      \item Un paquete para la creación de presentaciones en \LaTeX{}. \pause 
      \item Permite trabajar con los comandos de \LaTeX{}. \pause
      \item Adecuado para el uso de expresiones matemáticas y otros tales como:
      \begin{itemize}
        \item Secciones.\pause
        \item Expresiones matemáticas.\pause
        \item Imagenes.\pause
      \end{itemize}
    \end{itemize}
\end{frame}

\begin{frame}
	La gráfica de la función es la siguiente:
	\begin{center}	
%	\includegraphics[scale=•]{•}
	\end{center}
\end{frame}

\begin{frame}
    \frametitle{Otra diapositiva} 
	Para crear diapositivas con \LaTeX{} procedemos de la siguiente forma:
	\begin{enumerate}
	\item Indicar \textit{beamer} como clase de documento
	\item Especificar el tema a usar
	\item Seleccionar esquema de colroes
	\item Introducir \textbf{frames}
	\end{enumerate}
\end{frame}



\begin{frame}
    \frametitle{Otra diapositiva} \transduration{2}
\end{frame}

\begin{frame}
    \frametitle{Otra diapositiva} \transduration{0.5}
\end{frame}

\end{document}
