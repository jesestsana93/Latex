\documentclass[letterpaper,12pt]{article}

\usepackage[utf8]{inputenc} % Soporte para acentos
\usepackage[T1]{fontenc}    
\usepackage[spanish]{babel} % Español

%Paquetes para matemáticas
\usepackage{amsmath}		
\usepackage{amssymb} 		
\usepackage{amsfonts}
\usepackage{latexsym}

\usepackage{verbatim}

% Indicamos una separación entre los párrafos
\parskip=3mm

% Eliminamos la sangría de los párrafos
\parindent=0mm

%\pagestyle{myheadings}
%\markright{\LaTeX \hfill Fórmulas matemáticas \;\;}

\title{Fórmulas matemáticas}

\author{Autor del artículo}

\date{\today{}}

\begin{document}

\maketitle

\tableofcontents

\section{Entornos para expresiones matemáticas}

Para introducir contenido matemático podemos hacer uso del \textit{``modo matemático tipo texto''} o \textit{``modo texto''}:
%Para poner modos matemáticos de tipo texto o modos en línea hay dos opciones: $expresión$ o \(expresión\)
%Modo resaltado, centrado o extendido: $$expresión$$  o \[expresión\]
\verb@$ expresión $@ 

Este modo también se puede usar de la siguiente forma:

\verb@\( expresión \)@ ó \verb@\begin{math}...\end{math}@.

Otra posibilidad es hacer uso del \textit{``modo matemático extendido''} o \textit{``modo resaltado''}:

\verb@$$ expresión $$@

Este modo también puede usarse en la forma:

 \verb@\[ expresión \]@ ó \verb@\begin{displaymath} ... \end{displaymath}@

Existen diferencias entre estos dos entornos. Por ejemplo, con el modo texto podemos introducir una expresión matemática en medio de un párrafo: $x^2 + y^2 = cos(x^3)$.
Con el modo resaltado podemos introducir una expresión como la siguiente: $$ \int_{\alpha}^{\beta} x^{\alpha} - {\beta}^{x^2} $$ o bien, \[ \sum_{a}^{b} (1-x)\cdot a + (x-1)\cdot b \]

Cuando introducimos expresiones matemáticas en modo texto,$ \int_{\alpha}^{\beta} x^{\alpha} - {\beta}^{x^2} $	 es posible incluir el comando \verb@\displaystyle@ para que éstas se desplieguen en 
tamaño natural, de lo contrario \LaTeX{} las ajusta al tamaño del renglón. Por ejemplo: $ I_n=\int_{x=1}^{N} x^2 + 5 \, dx  $. 

Veamos la misma expresión en tamaño natural: $\displaystyle I_n=\int_{x=1}^{N} x^2 + 5 \, dx  $

\section{Sumas y restas}

Podemos introducir expresiones aritméticas simples, como las siguientes:

\[ x + y = 5 \]
%\cdot: punto centrado
\[ (x + y)\cdot a + (x-y)\cdot b = b - a \]

\section{Potencias, índices y subíndices}

También podemos introducir expresiones con potencias, índices y subíndices:

\subsection{Potencias e índices} 

$$ X^2 $$

Parte de las expresiones pueden estar dadas en términos de símbolos especiales, por ejemplo, podemos hacer
uso de $\infty$:

$$ X^\infty $$
$$ X^{-\infty} $$

$$ a^b $$

Si los exponentes están formados por varias expresiones deben colocarse entre llaves.

$$ x^{a + b} $$

\LaTeX{} proporciona expresiones para introducir algunas de las funciones más comunes:

$$ \sen^2(x) + \cos(x) $$

\subsection{Subíndices} 

$$ a_{1,1} + a_{1,2} + a_{1,3} $$

$$ x_1^2 $$ % es equivalente a $$x^2_1$$
$$x^{a^2}$$

$$ x_{1,2}^{(a + b)} $$

Cuando se introducen subíndices formados por expresiones de varios términos, conviene escribirlos 
de la siguiente forma:

$$ X_{_{N+1}} $$

Cómparemos esto con la siguiente expresión:

$$ X_{N+1} $$

Por medio de índices y subíndices se pueden introducir otras expresiones, por ejemplo integrales definidas
o sumatorias. 

Para desplegar el símbolo de integral podeos usar la expresión \verb@\int@. Para obtener el símbolo de la
sumatoria podemos usar \verb@\sum@. Veamos un par de ejemplos:

$$ \int_{-1}^{1} x^n \, dx $$

$$ \sum_{n=1}^{80} \, i^n $$

Este tipo de expresiones se revisaran más abajo.

\section{Raices}

Las raices cuadradas se pueden desplegar de la siguiente forma:

$$\sqrt{2 x + 5}$$
$$\sqrt{\sqrt{2 x} + \sqrt{x^5}}$$

También se pueden incluir raices n-ésimas:

$$\sqrt[5]{-b} \pm \sqrt[3]{x_1 + \sqrt{2 x -5}}$$
$$\sqrt[5]{-b} \mp \sqrt[3]{x_1 + \sqrt{2 x -5}}$$

\section{Fracciones}

Este tipo de expresiones se pueden introducir mediante los comandos: \verb@\over@, \verb@\frac{}{}@, o bien \verb@{ \atop }@. Por ejemplo:

$${x+2 \over x-3}$$

\[ \frac{x^2 + 1}{x-1}\]

\[ {{x + 1}\over{3}} \over {x+2} \]

$${\left( x + {1 \over x} \right)^{n+1 \over x}}$$

También se puede hacer uso de \verb@\cfrac@ y \verb@\dfrac@.

$\frac{1}{a} \,\, \cfrac{1}{a} \,\, \dfrac{1}{a}$

\section{Expresiones de dos niveles}

El comando \verb@\atop@ nos permite introducir expresiones como las que se muestran a continuación:

$\displaystyle {x^2-2 \atop x^3-1}$

$${x^2-2 \brace x^3-1}$$

$\displaystyle {x^2-2 \brack x^3-1}$

$${a \stackrel{f}{\rightarrow} b}$$
%En modo matematico, el UTF8 para los acentos no se reconoce por lo que hay que ponerlo de la manera tradicional
$${a \stackrel{f}{\leftarrow} b}$$

%$${a \stackrel{f}{\diagarrow} b}$$

$\sum_{\substack{0<i<n\\0<j<m}} x^i + y^j$
$$\sum_{\substack{0<i<n\\0<j<m}} x^i + y^j$$

\[ \prod_{\overset{i=0}{i\neq k}}^{n}\frac{w_i}{(w_i-w_k)} \]

$\sum_{i=1}^{n}	$
\section{Sumas y productos}

Los comandos \verb@\sum@ y \verb@prod@ nos permiten introducir sumas y productos en \LaTeX{}. Por ejemplo:

$$\sum a_i$$

$\displaystyle \sum_{i=1}^n a_i$

$$\sum_{A} a_i = \sum_{i=1}^n a_i $$

$$\sum_{A} a_i = \sum_{\substack{i=1\\j=0}}^n a_i $$

Algunos ejemplos de productos:

$$X= \prod_{i=1}^n x_i$$

$\displaystyle \prod_{A} a_i$

\section{Límites}

Para incluir expresiones que involucran límites podemos hacer uso del comando \verb@\lim@, de la siguiente forma:

$\displaystyle \lim f(x)$

Si queremos indicar el punto alrededor del cual se calculará el límite, podemos incluirlo de la siguiente forma:

$\displaystyle \lim_{x\to 0} f(x) $

Para indicar un límite lateral el símbolo se incluye como índice:

\[ \lim_{x \rightarrow 0^{+}} f(x) \]

\[ \lim_{x \rightarrow \infty^{+}} f(x) \]

\[ \lim_{x \to -\infty} f(x) \]

\section{Integrales y derivadas}

Las integrales se pueden introducir mediante el comando \verb@\int@. Por ejemplo:

$$\int \! f(x) \, \mathrm{d}x $$

En los modos matemáticos los espacios en blanco son eliminados, éstos deben introducirse por medio de las expresiones:

\verb@\thinspace o \, - 1.82pt@

\verb@\medspace o \: - 2.43pt@

\verb@\thickspace o \; - 3.04pt@

\verb@\quad - 10.95pt@

\verb@\qquad - 21.9pt@

\verb@\thinspace o \! - 1.82pt@

Veamos otros ejemplos:

$\displaystyle \int_a^b f(x) \, \mathrm{d}x $

\[ \int_a^b \int_c^d {f(x,y) \, \mathrm{d}y} \, \mathrm{d}x \]
{
En el caso de las derivadas y las parciales, las podemos expresar de la siguiente forma:

$$\frac{dy}{dx} f$$
Para una parcial:
$$\partial$$

\section{Delimitadores}

\LaTeX{} proporciona varios símbolos para delimitar expresiones. Por ejemplo:

$$ \left( {x^2}{y^3} \right) $$

$$ \left( \frac{x^2}{y^3} \right) $$

\[ \left\{ \frac{x^2}{y^3} \right\} \]

\[ \left( A=2 \middle| \frac{A^2}{B}>4 \right) \]

\[ x = a_0 + \cfrac{1}{a_1 + \cfrac{1}{a_2 + \cfrac{1}{a_3 + \cfrac{1}{a_4} } } } \]

\[ \left( x = a_0 + \cfrac{1}{a_1 + \cfrac{1}{a_2 + \cfrac{1}{a_3 + \cfrac{1}{a_4} } } } \,\, \right) \]

Otros ejemplos:

$\displaystyle \left[{x+1 \over (x-1)^2} \right]^n$

$$\left \langle \frac{a}{b} \right \rangle $$

$\displaystyle \left | \frac{a}{b} \right | $

\[ \left | x = a_0 + \cfrac{1}{a_1 + \cfrac{1}{a_2 + \cfrac{1}{a_3 + \cfrac{1}{a_4} } } } \,\, \right | \]

\[ \left \| x = a_0 + \cfrac{1}{a_1 + \cfrac{1}{a_2 + \cfrac{1}{a_3 + \cfrac{1}{a_4} } } } \,\, \right \| \]

\[ \left( x = a_0 + \cfrac{1}{a_1 + \cfrac{1}{a_2 + \cfrac{1}{a_3 + \cfrac{1}{a_4} } } } \,\, \right | \]

\[ f(x) = \left\{ a_0 + \cfrac{1}{a_1 + \cfrac{1}{a_2 + \cfrac{1}{a_3 + \cfrac{1}{a_4} } } } \right. \]

\section{Símbolos}

\LaTeX{} nos permite ingresar diversos tipos de símbolos matemáticos, entre otros podemos mencionar:

$\aleph$

$\delta$

$\pm$

$\subset${

$\approx$

$\in$

$\bigcap$

$\leqslant$

$\geqslant$

$\triangleright$

$\wedge$

$\diagdown$

$\longleftrightarrow$

$\Longleftrightarrow$

\begin{center}
\Large{\textbf{FIN}}
\end{center}

