\documentclass[letterpaper,12pt]{article}

\usepackage[utf8]{inputenc} % Soporte para acentos
\usepackage[T1]{fontenc}    
\usepackage[spanish,mexico]{babel} % Español

% Soporte de símbolos adicionales (matemáticas)
\usepackage{amsmath}		
\usepackage{amssymb}		
\usepackage{amsfonts}
\usepackage{latexsym}

\usepackage[lmargin=2cm,rmargin=2cm,top=2cm,bottom=2cm]{geometry}

\usepackage{tikz}
\usetikzlibrary{matrix}
\usetikzlibrary{shapes.geometric}

\usepackage{pgfplots}
%\pgfplotsset{compat=newest}

% Información para el título
\title{Figuras con TikZ 2}
\author{J. Luis Torres}

\begin{document}

\maketitle

\section{TikZ}

Poligonal:

\begin{tikzpicture}[scale=.9]
	\draw (1,0) -- (0,0) -- (0,1);
\end{tikzpicture}

\medskip 

Poligonal con rotación:

\begin{tikzpicture}[scale=.9]
	\draw[red, dashed, very thick, rotate=30] (1,0) -- (0,0) -- (0,1);
\end{tikzpicture}

\medskip 

Polígono:

\begin{tikzpicture}
	\draw (1,0) -- (0,0) -- (0,1) -- cycle;
\end{tikzpicture}


\medskip 

Segmentos perpendiculares:

\begin{tikzpicture}
	\draw (0,0) -| (1,1);
\end{tikzpicture}

\medskip 

\begin{tikzpicture}
	\draw (0,0) |- (1,1)|-(2,0);
\end{tikzpicture}

\medskip 

Curva de Bezier:

\begin{tikzpicture}
	\draw (0,0) .. controls (1,1) .. (4,0)
      (5,0) .. controls (6,0) and (6,1) .. (5,2);
\end{tikzpicture}

\medskip

\begin{tikzpicture}[scale=.7]
	\draw[line width=4pt] (0,0) to [out=90, in=180] (3,2) to [out=-90, in=90] (8,-2);
\end{tikzpicture}

\newpage

Flechas:

\begin{tikzpicture}
	\draw[->] (0,0) -- (5,0);

	\draw[dotted, >->>] (0,1) -- (5,1);

	\draw[|<->|] (0,2) -- (5,2);

	\draw[loosely dashed] (0,4) -- (5,4);

	\draw[densely dotted] (0,5) -- (5,5);

	\draw[->] (0,6) .. controls (1.2,-6.2) .. (5, 6);
\end{tikzpicture}

Nodos:

\begin{tikzpicture}[scale=.9, transform shape]
\tikzstyle{every node} = [circle, fill=gray!30]%Rellena de gris al 30%
\node (a) at (0, 0) {A};
\node (b) at +(0: 1.5) {B};
\node (c) at +(60: 1.5) {C};
\foreach \from/\to in {a/b, b/c, c/a}
\draw [->] (\from) -- (\to);
\end{tikzpicture}

\medskip

Texto sobre las flechas:

\begin{tikzpicture}
	\draw[->] (0,0) -- (2,0.5) node[pos=.5,sloped,above] {$x$};
	\draw[->] (0,0) -- (2,-.5) node[pos=.5,sloped,below] {$y$};
\end{tikzpicture}

Polígonos y figuras regulares:

\begin{tikzpicture}
	\draw (0,0) rectangle (1,1);
	\shade[top color=yellow, bottom color=black] (0,0) rectangle (2,-1);
	\filldraw[fill=green!20!white, draw=green!40!black] (0,0) rectangle (2,1);
\end{tikzpicture}

\begin{tikzpicture}
	\draw (0,0) circle [radius=1.5];
	\draw (0,0) circle (2cm);
	\draw (0,0) circle [x radius=1.5cm, y radius=10mm];
	\draw (0,0) circle [x radius=1.2cm, y radius= 8mm];
	\draw (0,0) circle [x radius=1cm, y radius=5mm, rotate=30];
\end{tikzpicture}

\begin{tikzpicture}
\node[draw=black,minimum size=5cm,regular polygon,regular polygon sides=5] (a) {};
\foreach \x in {1,2,...,5}
	\fill (a.corner \x) circle[radius=2pt];
\end{tikzpicture}

\begin{tikzpicture}
  \newdimen\R
  \R=0.8cm
  \draw (0:\R)
     \foreach \x in {72,144,...,360} {  -- (\x:\R) }
              -- cycle (360:\R) node[right] {5}
              -- cycle (288:\R) node[below right] {4}
              -- cycle (216:\R) node[below left] {3}
              -- cycle (144:\R) node[above left] {2}
              -- cycle  (72:\R) node[above right] {1};
\end{tikzpicture}

\begin{tikzpicture}
  \newdimen\R
  \R=0.8cm
  \draw [rotate=18] (0:\R)
     \foreach \x in {72,144,...,360} {  -- (\x:\R) }
              -- cycle (360:\R) node[right] {5}
              -- cycle (288:\R) node[below right] {4}
              -- cycle (216:\R) node[below left] {3}
              -- cycle (144:\R) node[above left] {2}
              -- cycle  (72:\R) node[above right] {1};
\end{tikzpicture}


\medskip

Arcos:

\begin{tikzpicture}
	\draw (0,0)  arc[radius = 8mm, start angle= 0, end angle= 270];
	\draw (0,0)  arc[x radius = 1.75cm, y radius = 1cm, start angle= 0, end angle= 315];
	\filldraw[fill=cyan, draw=blue] (0,0)  arc[radius = 15mm, start angle= 0, end angle= 270];
\end{tikzpicture}

\medskip

Funciones y curvas:

\begin{tikzpicture}
	\draw[help lines] (0,0) grid (2,3);
	\draw[step=0.5, gray, very thin] (-1.4,-1.4) grid (1.4,1.4);
	\draw (0,0) parabola (1,1.5) parabola[bend at end] (2,0);
	\draw (0,0) sin (1,1) cos (2,0) sin (3,-1) cos (4,0) sin (5,1);
\end{tikzpicture}

\medskip

\begin{tikzpicture}
\foreach \x in {0,...,9} 
  \draw (\x,0) circle (0.4);
\end{tikzpicture}

\medskip


\begin{tikzpicture}
\draw [help lines] (-2,0) grid (2,4); 
\draw [->] (-2.2,0) -- (2.2,0); 
\draw [->] (0,0) -- (0,4.2); 
\draw [green, thick, domain=-2:2] plot (\x, {4-\x*\x}); 
\draw [domain=-2:2, samples=50] plot (\x, {1+cos(pi*\x r});
\end{tikzpicture}

\medskip

\begin{tikzpicture}
  \matrix (m) [matrix of math nodes,row sep=3em,column sep=4em,minimum width=2em]
  {
     F_t(x) & F(x) & G(x)\\
     A_t & A & \\}; %Camino a conectar con lineas \path[-stealth]
  \path[-stealth] %edge para conectar los nodos
    (m-1-1) edge node [left] {$\mathcal{B}_X$} (m-2-1)
            edge [double] node [below] {$\mathcal{B}_t$} (m-1-2)
    (m-2-1) edge node [below] {$\mathcal{B}_T$}
            node [above] {$\exists$} (m-2-2)
    (m-1-2) edge node [right] {$\mathcal{B}_T$} (m-2-2)
            edge [dashed,-] (m-2-1)
    (m-2-2) edge [double] (m-1-3);
\end{tikzpicture}


\end{document}
