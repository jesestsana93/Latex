\documentclass{article}

\usepackage[T1]{fontenc}
\usepackage[spanish,mexico]{babel}
\usepackage[utf8]{inputenc}

\usepackage{tikz}

\begin{document}

\begin{tikzpicture}
\draw (0,0) -- (2,2); % segmento
\end{tikzpicture}

\begin{tikzpicture} % ejes
\draw[thin, blue] (0,2) -- (0,0) -- (2,0);
% segmento
\draw[line width =0.051cm, red, dashed]
(-1,-1) -- (1.5,1.5);
\end{tikzpicture}

\begin{tikzpicture}[scale=0.8] % Escalamiento de la figura 80%
\draw (-1,0) -- (4,0) node[right] {$x$}; % Ejes
\draw (0,-1) -- (0, 2) node[left] {$y$};
% Dominio: domain = a:b
\draw[smooth, domain = 0:2, color=red] plot (\x,\x)node[right] {$y = x$
};
%\x r indica que x se mide en radianes
\draw[smooth, domain = -2:2, color=blue] plot (\x,{sin(2*\x r)+1})
node[right] {$y = \sin(2*x)+1$};
\draw[smooth, domain = -1:1, color=black] plot (\x,{exp(\x)}) node[right
] {$y = e^x$};
\end{tikzpicture}

\begin{tikzpicture} % ejes
\draw[thin] (0,2) -- (0,0) -- (2,0);
% rectángulo borde negro, relleno rojo, esquina (0,0)
\draw[black, fill=red] (0,0) rectangle(1,1);
% círculo verde, centro =(1.5,.5) y radio 0.5cm
\draw[green, fill=green] (1.5,0.5) circle [radius
=0.5];
\end{tikzpicture}


\end{document}