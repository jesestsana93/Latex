% Otros paquetes que pueden generar este tipo de diagramas son: tikz-cd, pb-diagram y amscd. Éste ultimo puede mostrar problemas de compatibilidad con la opción spanish del paquete babel.
\documentclass[letterpaper,12pt]{article}

\usepackage[utf8]{inputenc} % Soporte para acentos
\usepackage[T1]{fontenc}    
\usepackage[spanish,mexico]{babel} % Español

% Soporte de símbolos adicionales (matemáticas)
\usepackage{amsmath}		
\usepackage{amssymb}		
\usepackage{amsfonts}
\usepackage{latexsym}

\usepackage[lmargin=2cm,rmargin=2cm,top=2cm,bottom=2cm]{geometry}

%\usepackage{tikz}
\usepackage[all]{xy}

% Información para el título
\title{Diagramas}
\author{J. Luis Torres}

\begin{document}

\maketitle

El paquete \textit{XY-pic} nos permite crear diagramas en \LaTeX. La forma de construirlos es usando una notación similar a la de las matrices. Por ejemplo:

\bigskip

\xymatrix{
      A & B \\
      C & D
    }
    
\bigskip

Para agregar flechas entre los elementos del diagrama, podemos hacer uso de la instrucción \verb@\ar@ y las letras \texttt{l} (left), \texttt{r} (right), \texttt{u} (up) y \texttt{d} (down). Con estos caracteres, o una combinación de ellos, podemos indicar la posición de las flechas. Por ejemplo:

\bigskip

\begin{equation*}
\xymatrix{
    A \ar[r] \ar[d] \ar[rd] & B \ar[d] \\
    C \ar[r] & D
}
\end{equation*}

\bigskip

En este caso, la expresión ``\verb@A \ar[r] \ar[d] \ar[rd]@'' indica que, a partir del elemento \texttt{A}, debe colocarse una flecha hacia el elemento de la derecha (\verb@\ar[r]@), una segunda flecha hacia el elemento de abajo (\verb@\ar[r]@) y una flecha hacia el elemento que se encuentra a la derecha, abajo (\verb@\ar[rd]@). Como se muestra en este último ejemplo, la forma de centrar este tipo de diagramas se puede hacer uso del entorno \texttt{equation}.

También podemos agregar texto a las flechas, incluyendo éste como índice o subíndice de la expresión ``\verb@\ar@''. Por ejemplo:

\bigskip

\begin{equation*}
\xymatrix{
    A \ar[r]^f \ar[d]_\phi \ar[rd] & B \ar[d]^\psi \\
    C \ar[r]_g & D
}
\end{equation*}

\bigskip

Existen muchos tipos de flechas disponibles, entre las que se encuentran:

\begin{verbatim}
  @{.>}
  @{-->}
  @{=>}
  @{=}
  @{^{(}->}
  @{->>}
  @{|->}
  @{~>}
\end{verbatim}    

Por ejemplo:

\bigskip

\begin{equation*}
\xymatrix{
    A \ar@{^{(}->}[r]^f \ar@{.>}[d]_\phi \ar@{=>}[rd] & B \ar@{=}[d]^\psi \\
    C \ar@{->>}[r]_g & D
}
\end{equation*}

\bigskip

Se pueden introducir flechas curvas incluyendo las expresiones: ``@\verb@/^/@'' ó ``@\verb@/_/@''.

\bigskip

\begin{equation*}
\xymatrix{ A \ar[r] & B \ar@/_/@{->}[l] }
\end{equation*}

\bigskip

\begin{equation*}
\xymatrix{
    T \ar@/^/[rrd] \ar@/_/[rdd] \ar@{.>}[rd] \\
    & A \ar[r] \ar[d] & B \ar[d]_\beta \\
    & C \ar[r] & D \ar@/_/[u]_\alpha
  }
\end{equation*}

\bigskip

Podemos agregar una distancia que se agregará a la flecha para alejarla de una parte del diagrama:

\bigskip

\begin{equation*}
\xymatrix{
    T \ar@/^/[rrd] \ar@/_/[rdd] \ar@{.>}[rd] \\
    & A \ar[r] \ar[d] & B \ar[d]_\beta \\
    & C \ar[r] & D \ar@/_/[uull]_\alpha
  }
\end{equation*}

\bigskip

En el siguiente diagrama incluimos la expresión ``\verb@|!{[ur];[dr]}\hole@'' para indicar que una de las flechas debe mostrarse con dos segmentos de recta, para indicar que ésta se encuentra detras de otra de las flechas:

\bigskip

\begin{equation*}
\xymatrix {
    A \ar[rr] \ar[rd] \ar[dd] && B \ar[dd] \\
    & C \ar[ru] \ar[dd] \\
    A' \ar[rr] |!{[ur];[dr]}\hole \ar[rd] && B' \\
    & C' \ar[ru] 
    }
\end{equation*}

\bigskip

Podemos hacer lo mismo para las flechas verticales, incluyendo la expresión:

``\verb@|!{[dl];[dr]}\hole@'':

\bigskip

\begin{equation*}
\xymatrix {
    A \ar[rr] \ar[dd] \ar[dr] && B \ar[dr] \ar[dd] |!{[dl];[dr]}\hole \\
    & A \ar[rr] \ar[dd] && B' \ar[dd] \\
    C \ar[rr] |!{[ur];[dr]}\hole \ar[dr] && D \ar[rd] \\
    & C' \ar[rr] && D' \\
  }    
\end{equation*}

\bigskip

En el siguiente diagrama colocamos flechas que apuntan al mismo elemento en el que se originan:

\bigskip

\begin{equation*}
\xymatrix{
    A \ar@(ul,dl)[] \ar@/^/[r] & 
    B \ar@(ur,dr)[] \ar@/^/[l] 
}
\end{equation*}
 
\bigskip

En este ultimo ejemplo usamos la expresión ``\texttt{@(salida, entrada)}'', en donde salida y entrada pueden ser \texttt{u}, \texttt{d}, \texttt{r}, \texttt{r}, \texttt{ur}, \texttt{ud}, \texttt{ul}, \texttt{dl}, éstas se usan para indicar en que dirección debe salir la flecha y en que dirección debe entrar en su destino. Por ejemplo:

\bigskip

\begin{equation*}
\xymatrix{A\ar@(ul,dl)[]\ar@(ur,r)[r] &B}
\end{equation*}

\end{document}
