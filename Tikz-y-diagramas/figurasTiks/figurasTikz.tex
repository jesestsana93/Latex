\documentclass[letterpaper,12pt]{article}

\usepackage[utf8]{inputenc} % Soporte para acentos
\usepackage[T1]{fontenc}    
\usepackage[spanish,mexico]{babel} % Español

% Soporte de símbolos adicionales (matemáticas)
\usepackage{amsmath}		
\usepackage{amssymb}		
\usepackage{amsfonts}
\usepackage{latexsym}

\usepackage[lmargin=2cm,rmargin=2cm,top=2cm,bottom=2cm]{geometry}

\usepackage{tikz}

\usepackage{pgfplots}
%\pgfplotsset{compat=newest}

% Información para el título
\title{Figuras con TikZ}
\author{J. Luis Torres}

\begin{document}

\maketitle

\section{TikZ}

El paquete  \textit{tikz} nos permite crear figuras en documentos de \LaTeX{}. Para poder llevar a cabo esto podemos hacer uso del entorno \textbackslash begin\{tikzpicture\} \textbackslash end\{tikzpicture\} o bien \textbackslash tikz[opciones]\{comandos de tikz\}.

\medskip%Salto de línea entre la figura y el texto

Veamos algunos ejemplos:

\medskip

Una línea:

\medskip

%Entorno para crear gráficos
\begin{tikzpicture}
%\draw(primer punto) linea (segundo punto"10 cm a la derecha y 20 puntos arriba);
\draw (0,0) --(10cm,20pt);
\end{tikzpicture}

\begin{tikzpicture}
%\draw(primer punto) linea (segundo punto"10 cm a la derecha y 20 puntos arriba);
\draw (1,0) --(2cm,4cm)--(4cm,-3cm)--(5,2)--cycle;
%Con cicle indicamos que es una figura cerrada
\end{tikzpicture}
\medskip

Una poligonal:

\medskip

\begin{figure}[h!]
	\centering
	\begin{tikzpicture}
		\draw (0,0) --(1,2) -- (2,3) -- (1,0);
		%Genera la poligonal y un conjunto de "líneas de ayuda"
		\draw[help lines] (0,0) grid (2,3);
	\end{tikzpicture}
\end{figure}

La misma, aplicando una escala:

\begin{figure}[h!]
	\centering
	\begin{tikzpicture}[scale=3]
		\draw (0,0) --(1,2) -- (2,3) -- (1,0);
		\draw[help lines] (0,0) grid (2,3);
	\end{tikzpicture}
	\caption{Ejemplo de TikZ}
\end{figure}

\newpage

Flechas:

\begin{figure}[h!]
	\centering
	\begin{tikzpicture}
		\draw [->] (0,0) -- (2,0);
		\draw [<-] (0, -0.5) -- (2,-0.5);
		\draw [|->] (0,-1) -- (2,-1);
	\end{tikzpicture}
\end{figure}

\begin{figure}[h!]
	\centering
	\begin{tikzpicture}
		\draw [->] (-2,0) -- (5,0);
		\draw [->] (0,-2) -- (0,5);
	\end{tikzpicture}
\end{figure}

Líneas de diferente ancho:

\begin{figure}[h!]
	\centering
	\begin{tikzpicture}
		\draw [ultra thick] (0,1) -- (2,1);
		\draw [thick] (0,0.5) -- (2,0.5);
		\draw [thin] (0,0) -- (2,0);
		\end{tikzpicture}
\end{figure}

\begin{figure}[h!]
	\centering
	\begin{tikzpicture}
		\draw [->][ultra thick] (0,1) -- (2,1);
		\draw [thick] (0,0.5) -- (2,0.5);
		\draw [thin] (0,0) -- (2,0);
		\end{tikzpicture}
\end{figure}

\medskip

Líneas punteadas y en segmentos:

\begin{figure}[h!]
	\centering
	\begin{tikzpicture}
		\draw [dashed, ultra thick] (0,1) -- (2,1);
		\draw [dashed] (0, 0.5) -- (2,0.5);
		\draw [dotted] (0,0) -- (2,0);
	\end{tikzpicture}
\end{figure}

Colores:

\begin{figure}[h!]
	\centering
	\begin{tikzpicture}
		\draw [gray] (0,1) -- (2,1);
		\draw [red] (0, 0.5) -- (2,0.5);
		\draw [<->, dashed, blue] (0,0) -- (2,0);
	\end{tikzpicture}
\end{figure}

Algunos de los colores predeterminados son: red , green , blue , cyan, magenta , yellow , black , gray , darkgray , lightgray ,
brown , lime , olive , orange , pink , purple , teal , violet y blanco

Este tipo de elementos se pueden colocar dentro de un texto: \begin{tikzpicture} \draw [yellow, line width=6] (0,0) -- (.5,0); \end{tikzpicture}, no
es necesario colocarlos de forma independiente.

Otro tipo de figuras:

\begin{figure}[h!]
	\begin{tikzpicture}
		\draw [blue] (0,0) rectangle (1.5,1);
	\end{tikzpicture}
\end{figure}

\begin{figure}[h!]
	\begin{tikzpicture}
		\draw [red, ultra thick] circle [radius=0.5];
	\end{tikzpicture}
\end{figure}

\begin{figure}[h!]
	\begin{tikzpicture}
		\draw [gray] (0,0) arc [radius=1, start angle=45, end angle= 120];
	\end{tikzpicture}
\end{figure}

Funciones:

\begin{figure}[h!]
	\begin{tikzpicture}[xscale=13,yscale=3.8]
		\draw [<->] (0,0.8) -- (0,0) -- (0.5,0); %plot(variable, funcion) domain=indica que x va de 0 0.5 
		\draw[green, ultra thick, domain=0:0.5] plot (\x, {0.025+\x+\x*\x});
\end{tikzpicture}\end{figure}


\newpage

Imagenes con relleno:

\begin{figure}[h!]
	\begin{tikzpicture}
		\draw [fill=red,ultra thick] (0,0) rectangle (1,1);
		\draw [fill=red,ultra thick,red] (2,0) rectangle (3,1);
		\draw [red, fill=blue] (4,0) -- (5,1) -- (4.75,0.15) -- (4,0);
		\draw [fill] (7,0.5) circle [radius=0.1];
		\draw [fill=orange] (9,0) rectangle (11,1);
		\draw [fill=white] (9.25,0.25) rectangle (10,1.5);
\end{tikzpicture}\end{figure}

También podemos incluir texto:

\begin{figure}[h!]
	\begin{tikzpicture}
		\draw [thick, <->] (0,2) -- (0,0) -- (2,0);
		\node at (1,1) {yes};
		\draw[help lines] (0,0) grid (2,2);
	\end{tikzpicture}
\end{figure}

El texto lo podemos ubicar en otra posición:

\begin{figure}[h!]
	\begin{tikzpicture}
		\draw [thick, <->] (0,2) -- (0,0) -- (2,0);
		\node [above] at (1,1) {yes};
		\draw[help lines] (0,0) grid (2,2);
	\end{tikzpicture}
\end{figure}

\begin{figure}[h!]
	\begin{tikzpicture}
		\draw [thick, <->] (0,2) -- (0,0) -- (2,0);
		\node [below] at (1,1) {$x^2+2x-1$};
		\draw[help lines] (0,0) grid (2,2);
	\end{tikzpicture}
\end{figure}

\begin{figure}[h!]
\centering
\begin{tikzpicture}
	\draw (0,2)--(2,4)--(3,2)--(2,0);
	\draw (3,2)--(4,0);
	\draw [fill=white](2,4) circle [radius=0.25];%Primero el circulo y despues la letra
	\node at (2,4) {A};
	\draw [fill=white](3,2) circle [radius=0.25];
	\node at (3,2) {C};
		\node at (4,2.5) {Nodo C};
	\draw [fill=white](2,0) circle [radius=0.25];
	\node at (2,0) {D};
	\end{tikzpicture}
\caption{Arbol binario}
\end{figure}

\newpage

\begin{tikzpicture} 
	\begin{axis}[ 
	     xlabel=$x$,
		 ylabel={$f(x) = x^2 - x +4$}
		] 
		\addplot {x^2 - x +4};
	\end{axis} 
\end{tikzpicture}

\end{document}
