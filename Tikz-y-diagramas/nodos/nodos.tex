% Otros paquetes que pueden generar este tipo de diagramas son: tikz-cd, pb-diagram y amscd. Éste ultimo puede mostrar problemas de compatibilidad con la opción spanish del paquete babel.
\documentclass[letterpaper,12pt]{article}

\usepackage[utf8]{inputenc} % Soporte para acentos
\usepackage[T1]{fontenc}    
\usepackage[spanish,mexico]{babel} % Español

% Soporte de símbolos adicionales (matemáticas)
\usepackage{amsmath}		
\usepackage{amssymb}		
\usepackage{amsfonts}
\usepackage{latexsym}

\usepackage[lmargin=2cm,rmargin=2cm,top=2cm,bottom=2cm]{geometry}

\usepackage{tikz}

% Información para el título
\title{Diagramas}
\author{J. Luis Torres}

\begin{document}

\maketitle

\begin{tikzpicture}
	\tikzstyle{every node}=[draw,shape=circle];
	\node (v0) at (0:0) {$v_0$};
	\node (v1) at ( 0:1) {$v_1$};
	\node (v2) at ( 72:1) {$v_2$};
	\node (v3) at (2*72:1) {$v_3$};
	\node (v4) at (3*72:1) {$v_4$};
	\node (v5) at (4*72:1) {$v_5$};

	\draw (v0) -- (v1)
		(v0) -- (v2)
		(v0) -- (v3)
		(v0) -- (v4)
		(v0) -- (v5);
\end{tikzpicture}

\begin{tikzpicture}
	\tikzstyle{every node}=[draw,shape=circle];
	\node (v0) at (3,3) {$v_0$};
	\node (v1) at (2,2) {$v_1$};
	\node (v2) at (4,2) {$v_2$};
	\node (v3) at (1,1) {$v_3$};
	\node (v4) at (3,1) {$v_4$};
	\node (v5) at (5,1) {$v_5$};

	\draw (v0) -- (v1)
		(v1) -- (v3)
		(v0) -- (v2)
		(v2) -- (v4)
		(v2) -- (v5);
\end{tikzpicture}


\end{document}
