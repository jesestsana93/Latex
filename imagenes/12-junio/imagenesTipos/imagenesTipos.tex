% http://en.wikibooks.org/wiki/LaTeX/Importing_Graphics

% latex -> \usepackage[dvips]{graphicx}
% De esta forma podemos insertar archivos .eps

% pdflatex -> 	\usepackage[pdftex]{graphicx}
%				\usepackage{epstopdf}
% De esta forma podemos insertar: .jpg, .png, .pdf y .eps; éste último requiere epstopdf

\documentclass[letterpaper,12pt]{article}

\usepackage[utf8]{inputenc} % Soporte para acentos
\usepackage[T1]{fontenc}    
\usepackage[spanish,mexico]{babel} % Español

% Para inserción de imagenes, compilando con pdflatex con imagenes en formato
%.jpg,.png,.eps, .pdf
%usepackage[pdftex]{graphicx}
% El siguiente paquete permite la inclusión de archivos .eps, compilando con pdflatex
%\usepackage{epstopdf}

% Para inserción de imagenes, compilando con latex con los siguientes formatos:
%.ps, .eps
\usepackage[dvips]{graphicx}

\usepackage{verbatim}

\usepackage[lmargin=2cm,rmargin=2cm,top=2cm,bottom=2cm]{geometry}

% Información para el título
\title{Imagenes de mapa de bits y vectoriales}
\author{J. Luis Torres}
\date{12 de junio de 2015}

\begin{document}


\maketitle

Ejemplos de figuras insertadas en un documento de \LaTeX{}.

%Dependiendo de la imagen a utilizar el compilador sera diferenteperritosperritosperritosperritos

%Entorno contenido en el paquete verbatim, para hacer un comentario multilinea
%Es decir, comentar bloques de lineas para que no se compilen
\begin{comment}
\begin{figure}[h!]
\centering
%\includegraphics[scale=2]{Floral.jpg}
\includegraphics{Floral2.jpg}
\caption{Mapa de bits.}
\end{figure}
\end{comment}

%\begin{comment}
\begin{figure}[h!]
\centering
%\includegraphics[trim= 10px 50px 20px 5px,clip,scale=0.5]{Floral.eps}
\scalebox{2}{\includegraphics*[50,660][130,740]{Floral.eps}}
\caption{Vectorial.}
\end{figure}
%end{comment}

\end{document}
