\documentclass[letterpaper,12pt,twoside]{book}

%\newcommand{Nombre}{Expresion}
\newcommand{\hola}{\textit{\textbf{\Huge{Hola}}}}

\usepackage[utf8]{inputenc} 
\usepackage[T1]{fontenc}    
\usepackage[spanish,mexico]{babel} 

\usepackage{amsmath}		
\usepackage{amssymb}		
\usepackage{amsfonts}
\usepackage{latexsym}

\usepackage{graphicx}

% Información para el título
\title{Cambiar Nombres de Elementos del Documento}
\author{J. Luis Torres}

% El comando "renewcommand permite cambiar los nombres usados por LaTeX para algunos de los elementos del
% documento. Su sintaxis es:
%
% \renewcommand{comando}{Nuevo nombre}
%
% Algunos de los comandos más comunes son:
%
% Comando:			Valor predeterminado:
% \abstractname			Resumen
% \appendixname			Apéndice
% \bibname				Bibliografía (clases book y report)
% \refname				Referencias  (clase artículo)
% \chaptername			Capítulo
% \contentsname			Índice general
% \figurename			Figura
% \listfigurename		Índice de Figuras
% \listtablename		Índice de Tablas
% \pagename				Página
% \partname				Parte
% \tablename			Tabla

% Por otro lado, podemos cambiar la numeración de las páginas con el siguiente comando:
% \setcounter{page}{número}
% También podemos especificar el tipo de números a usar con el comando:
% \pagenumbering{estilo}
% Donde "estilo" puede ser: 
% arabic - Números arábigos
% roman - Números romanos en minúsculas.
% Roman - Números romanos en mayúsculas.
% alph - Numeración alfabética con minúsculas.
% Alph - Numeración alfabética con mayúsculas. 

\begin{document}

%\renewcommand{\chaptername}{Contenido}
% Por ejemplo:
%Para comentar varios bloques, selecciono la parte a comentar mas Cntrl t
\renewcommand{\contentsname}{Lista de Temas}
\renewcommand{\partname}{Bloque}
\renewcommand{\chaptername}{Tema}
\renewcommand{\appendixname}{Complemento}
\renewcommand{\figurename}{Imagen}
\renewcommand{\listfigurename}{Lista de Imagenes}
\renewcommand{\tablename}{Cuadro}
\renewcommand{\listtablename}{Lista de Cuadros}
\renewcommand{\bibname}{Libros}

\maketitle
\thispagestyle{empty}

\frontmatter
\setcounter{page}{1}

\tableofcontents

\listoffigures
\addcontentsline{toc}{chapter}{Lista de Imagenes}

\listoftables
\addcontentsline{toc}{chapter}{Lista de Cuadros}

\mainmatter

\part{Elementos Flotantes}

\chapter{Imagenes}

\hola

Escudo de la Facultad de Ciencias:

\begin{figure}[h!]
\centering
\includegraphics[scale=0.5]{escudoCiencias}
\caption{Facultad de Ciencias}
\end{figure}

\chapter{Tablas}

\LaTeX{} soporta diferentes tipos de tablas.

\section{Tabla Simple}

El siguiente es un ejemplo sencillo de tabla:

\begin{table}[h!]
	\centering
	\begin{tabular}{|c|c|c|}
		\hline 
		$\alpha$ & $\beta$ & $x^2+y^2$ \\ 
		\hline 
		76.267 & \LaTeX & $\frac{2x}{\gamma}$ \\ 
		\hline 
		36 & hola & encabezados \\ 
		\hline 
	\end{tabular} 
	\caption{Tabla de datos}
\end{table}

\part{Expresiones Matemáticas}

\chapter{Matemáticas}

Capítulo de expresiones matemáticas.

\section{Expresiones matemáticas}

\[ f(x) = \left\{ a_0 + \cfrac{1}{a_1 + \cfrac{1}{a_2 + \cfrac{1}{a_3 + \cfrac{1}{a_4} } } } \right. \]

$$\sum_{A} a_i = \sum_{i=1}^n a_i $$

\appendix

\chapter{Símbolos}
\pagenumbering{roman}
\setcounter{page}{1}

\LaTeX{} nos permite ingresar diversos tipos de símbolos matemáticos, entre otros podemos mencionar:

$\aleph$

$\delta$

$\pm$

$\subset${

$\approx$

$\in$

$\bigcap$

$\leqslant$

$\geqslant$

$\triangleright$

$\wedge$

$\diagdown$

$\longleftrightarrow$

$\Longleftrightarrow$

\begin{thebibliography}{99}

\bibitem{spivakCalc} Spivak, Michael; Calculus; Reverte; 1996.

\bibitem{Lamport} Lamport, L.; \LaTeX{}; Addison-Wesley. 1996.

\end{thebibliography}
\addcontentsline{toc}{chapter}{Bibliografía}

\end{document}
