\documentclass{article}

\usepackage[T1]{fontenc}
\usepackage[spanish,mexico]{babel}
\usepackage[utf8]{inputenc}

%Parte de teorema y definición
\usepackage{amsmath}
\newtheorem{unadefi}{Definición}[section] % Formato 3.1, 3.2
\newtheorem{unteo}{Teorema}[section] % Formato 3.1.1, 3.1.2


\usepackage[shortlabels]{enumitem}
\usepackage[x11names,table]{xcolor}%Parte de tablas



\begin{document}

\section{Ejemplos matemáticos}

\begin{center}
{\fboxsep 12pt
\fcolorbox {orange}{white}{
\begin{minipage}[t]{10cm}
$0^0$ es una expresión indefinida.
	$0^0$ es una expresión indefinida.
	Si $a>0$ entonces $a^0=1$ pero $0^a=0.$
	Sin embargo, convenir en que $0^0=1$ es adecuado para que
	algunas fórmulas se puedan expresar de manera sencilla,
	sin recurrir a casos especiales, por ejemplo
	$$e^x=\sum _{n=0}^{\infty }\frac{x^n}{n!}$$
	$$(x+a)^n=\sum_{k=0}^n \binom{n}{k}x^k a^{n-k}$$
\end{minipage}
}}
\end{center}

\[ \overbrace{(x_i-1)}^{K_i}f(x)+\underbrace{(x_i-1)}_{K_i}g(x)
= K_i(f(x)+g(x)) \]

$\pmb{\cos(x+2\pi)=\cos x}$

\begin{equation}
\log_{2}(xy)=\log_2x + \log_2y
\end{equation}

%El entorno itemize usa puntos u otros símbolos para los items mientras que description permite
%descriptores con texto
\begin{description}
	\item[Media muestral:] $\frac{1}{n-1}\sum_{i=1}^n (X_i-\bar{X_n})^2$
	\item[Varianza muestral:] $\frac{1}{n-1} \sum_{i=1}^n (X_i-\bar{X_n})^2$
	\item[Momentos muestrales:] $\frac{1}{n} \sum_{i=1}^n X_i^k$
\end{description}

\section{Listas}

\begin{enumerate}[label=\emph{\alph*})]
	\item Uno\hrulefill Tiempo: 2:45 hrs
	\item Dos \dotfill Tiempo: 2:45 hrs
	\item tres \hfill Tiempo: 2:45 hrs
	\rule[0.5cm]{11cm}{0.01cm}%\rule[xcm]{ycm}{zcm} . Este comando se usa para dibujar una línea horizontal o vertical de
% ycm y grosor zcm. La distancia de la línea a la base del texto se controla con el primer parámetro [xcm].
\end{enumerate}

%Agregar un texto en negrita al item y correr margen izquierdo:
\begin{enumerate}[label=\textbf{Idea (\emph{\alph*})}, leftmargin=2cm]
	\item De nuevo Uno
	\item Dos
\end{enumerate}

\begin{enumerate}
	\item[\fbox{1.}] {\bf Procedimiento}{\em Aprendizaje}
	\item[\fbox{2.}] {\bf comienzo} %Descriptor personalizado
\end{enumerate}

\newpage
\begin{enumerate}[label=\emph{\Roman*})]
	\item Primer nivel (en Romanos)
	\item Nivel uno

	\begin{enumerate}[label=\emph{\arabic*})]
		\item Segundo nivel (en numeración arábiga)
		\item Nivel dos

		\begin{enumerate}[label=\emph{\alph*})]
			\item Tercer nivel (numeración alfabética)
			\item Nivel tres


			\begin{enumerate}[label=\emph{$\bullet$})]
				\item Cuarto nivel (usamos {\tt bullet})
				\item Nivel máximo
			\end{enumerate}
		\end{enumerate}
	\end{enumerate}	
\end{enumerate}

\section{Color en tablas}
\begin{table}[h!]
\centering
\rowcolors{1}{}{gray!20}
\begin{tabular}{ll}
\rowcolor{LightBlue2} $x_{n+1}$ & $|x_{n+1}-x_n|$\\ \hline
1.20499955540054 & 0.295000445\\
1.17678931926590 & 0.028210236\\
1.17650193990183 & 3.004$\times10^{-8}$\\
1.17650193990183 & 4.440$\times10^{-16}$\\ \hline
\end{tabular}
\caption{Iteracion de Newton para $x^2-\cos(x)-1=0$ con $x_0=1.5.$}
\end{table}

\begin{tabular}{ll}
\rowcolor{LightBlue2} $x_{n+1}$ & $|x_{n+1}-x_n|$\\ \hline
\cellcolor[gray]{0.80} 1.20499955540054 & 0.295000445\\
1.17678931926590 & 0.028210236\\
1.17650196994274 & 0.000287349\\
1.17650193990183 & 3.004$\times10^{-8}$\\
\cellcolor[gray]{0.80} 1.17650193990183 & 4.440$\times10^{-16}$\\ \hline
\end{tabular}

\section{Teoremas}

\begin{unadefi}
Sean $a,b$ enteros con $b \not = 0.$ Decimos que $b$ divide a $a$
si existe un entero $c$ tal que $a=bc.$
\end{unadefi}

\begin{unteo}
Si $a,b \in Z$ y si $a|b\;$ y $\;b|a$ entonces $\;|a|=|b|$
\end{unteo}


\end{document}