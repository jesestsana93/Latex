% Aquí comienza el preámbulo
\documentclass[letterpaper,12pt]{book}

%Como es un libro cada que termina un capitulo se deja una hoja en blanco
%Para determinar que clase hay que uilizar hay que tener en cuenta:
%section....article
%part.....book
%chapter....report
%Diapositivas....beamer
%Tripico....leatlef


% Este es un comentario

% Soporte para acentos. En Windows se puede usar latin1 o utf8, se recomienda utf8
\usepackage[utf8]{inputenc}
% Español, división de sílabas
\usepackage[spanish,mexico]{babel}
% Usar codificación T1
\usepackage[T1]{fontenc}

% Algunos paquetes que necesitaremos más adelante
\usepackage{multicol}
\usepackage{enumerate}

% Una forma de establecer los márgenes
%\topmargin = -2cm %margen superior
%\oddsidemargin= 0cm  %margen izquierdo
%caja de texto
%\textheight = 23cm
%\textwidth = 17cm


\renewcommand{\labelitemi}{$-$}
%\renewcommand{\labelitemi}{$\alpha$}


%Identación en la primera linea
%\parindent=0mm

\title{Mi primer libro en \LaTeX{}}
\author{Jesus Sanchez}
\date{10 de junio de 2015}


% Aquí termina el preámbulo e inmediatamente abajo inicia el cuerpo del documento
%%%%%%%%%%%
%%%%  Ninguna de las expresiones anteriores debe aparecer dentro del cuerpo del documento
%%%%%%%%%%%
\begin{document}

\maketitle
%Quita numero y encabezado a la pagina
\thispagestyle{empty}

\frontmatter %Parte inicial del libro(Agradecimientos,prólogo, prefacio)

\tableofcontents

\pagebreak


%chapter*{} no se enumera dicho "capitulo"
\chapter*{Prologo}
%Agregamos dicho "capitulo" que no esta enumerado a la tabla de contenidos, de otra manera no aparecera
%el prologo
%\addcontentsline {donde se agrega}{tipo de informacion a colocar}{Linea de texto a agregar}
%\addcontentsline{toc=table of contents}{chapter/section/subsection}{Prologo}
\addcontentsline{toc}{chapter}{Prologo}

Ejemplo de documento \LaTeX{} de la clase {\ttfamily book} con una estructura básica. 
Incluye secciones, subsecciones y referencias cruzadas. También se incluyen listas numeradas,
listas no numeradas y referencias cruzadas.

%Forma de poner una seccion
\addcontentsline{toc}{section}{Subprologo}

\mainmatter %Parte principal del libro(Capítulos del libro)
% Nuevo capitulo
\chapter{Formato de texto}

Veamos algunas otras maneras de dar formato al texto.

Podemos hacer uso de diferentes tipos de letra, por ejemplo:


\section{Tipo de fuentes}

\textrm{Este texto está en Romanas}

% Otra forma es {\rm }
{\rm Otro texto en romanas}

\smallskip

\emph{Enfática}
% Otra forma es {\em }

\smallskip

\textbf{Negritas}
% Otra forma es {\bf }

\medskip

\textit{itálicas}
% Otra forma es {\it }

\bigskip

\textsl{Slanted}
% Otra forma es {\sl }

\smallskip

\textsf{Sans Serif}
% Otra forma es {\sf }

\smallskip

\textsc{Small Caps}
% Otra forma es {\sc }

\smallskip

\texttt{Typewriter}
% Otra forma es {\tt }

\underline{Subrayado}

Estos tipos de letras se pueden combinar para producir, por ejemplo, letras itálicas negritas: \textbf{\textit{ejemplo}}.

\part{Fuentes}
% Nuevo capitulo
\chapter{Tamaño de letras}

Al igual que con los estilos de la fuente, tambien podemos especificar un tamaño para las letras, basado en el tamaño de la fuente que utiliza el documento actual.

Una forma de especificar estos tamaños es como se muestra a continuacion:

% Colocar esto en una lista no numerada
{\tiny tiny} %para la fuente mas pequeña se usa tiny
% también se puede utilizar \begin{tiny} ... \end{tiny}

{\scriptsize scriptsize}
% también se puede utilizar \begin{scriptsize} ... \end{scriptsize}

{\small small}
% también se puede utilizar \begin{small} ... \end{small}

{\normalsize normal}

{\large large}
% también se puede utilizar \begin{large} ... \end{large}

{\Large Large}
% también se puede utilizar \begin{Large} ... \end{Large}

{\LARGE LARGE}
% también se puede utilizar \begin{tiny} \end{tiny}

{\huge huge}
% también se puede utilizar \begin{huge} ... \end{huge}

{\Huge Huge}
% también se puede utilizar \begin{Huge} ... \end{Huge}

Todos estos tamaños están basados en el tamaño de la fuente del documento.

Estos tamaños se pueden combinar con los estilos de la sección anterior, por ejemplo:

%Entorno
\begin{Large}
\textbf{\textit{Ojalá aprenda algo de \LaTeX.}}
\end{Large}

% Nuevo capitulo
\chapter{Texto centrado}

Se puede indicar que una parte del documento aparezca centrado en la caja de texto mediante el siguiente ambiente:

\begin{center}
\LARGE{\textbf{\textit{\LaTeX es \\ divertido!!!}}}
\end{center}

% Nuevo capitulo
\chapter{Texto alineado a la derecha}

La siguiente expresión nos permite colocar texto alineado a la derecha:

\hfill Este texto aparecerá alineado al margen derecho.

En este caso solamente alineamos a la derecha \hfill una parte.

Podemos incluir una linea o puntos de la siguiente forma:

Sección 1 \hrulefill Página 17

Sección 2 \dotfill Página 24

% Nuevo capitulo
\chapter{Espacios verticales y horizontales}

Los espacios horizontales se pueden introducir mediante la expresión:

\hspace{2in}Dos pulgadas.

\hspace{3cm}Tres centimetros.
\hspace{3.5cm}

\hspace{50pt}Cincuenta puntos.

\hspace{3em} em

\hspace{-3cm}En el margen.

$x_{2}$ \hspace{4cm} $x^{2}$

Los espacios verticales se pueden introducir de la siguiente forma:

\vspace{2cm}Texto ubicado dos centimetros abajo.

%\vspace{-5cm} \hspace{6cm} Texto ubicado 5 cm arriba y 6 a la derecha.

\vspace{3cm}

\newpage
% Nuevo capitulo
\chapter{Uso de columnas}

La instruccion \textit{multicols} nos permite introducir texto en columnas. Para poder hacer uso de esto es necesario cargar el paquete \textbf{multicol}. Veamos un ejemplo:

% Colocar en 2 columnas
\begin{multicols}{4}
Este es  un ejemplo de texto en dos columnas, puede notarse que \LaTeX{} se encarga de distrobuir el texto de tal forma que las columnas se muestren de igual tamaño. De la misma forma se puede introducir texto en tres o más columnas.\\
Para el caso de dos columnas se puede hacer uso de \textbf{twocolumn}.\\
También se puede indicar el tamaño de cada una de las columnas, pero para esto es necesario hacer uso del ambiente \textbf{\textbackslash minipage} 
o de \textbf{\textbackslash parbox}.
% fin de las tres columnas
\end{multicols}

% Nueva pagina
\pagebreak

% Nuevo capitulo
\chapter{Notas al pie}

Las notas al pie se pueden introducir mediante el comando \textit{footnote}:

Este comando debe colocarse en el lugar en el que aparecerá la referencia a la nota. \footnote{La dirección del sitio oficial de \LaTeX{} es http://www.ctan.org/}.

% Nuevo capitulo
\chapter{Texto Verbatim}
Texto Verbatim \footnote{Verbatim es un entorno en el que no se interpretan los símbolos.}


El entorno \textbf{verbatim} nos permite introducir un bloque de texto en el que se va a respetar la forma del texto y los espacios. Por ejemplo:

% Colocar en verbatim
%Todo lo que esta dentro de aqui no se interpreta en este entorno
\begin{verbatim}
Este texto       debe aparecer   exactamente como lo escribimos.

Se deben respetar los espacios y

saltos de linea. \textbf{negritas} 
\textit{itálicas}
\end{verbatim}
% fin de verbatim

El uso más común de este entorno es para introducir código fuente de algún lenguaje de programación.

% Nuevo capitulo
\chapter{Listas numeradas}
Este tipo de listas se pueden introducir mediante el entorno \textit{enumerate}

\begin{enumerate}
	\item Primer paso
	\item Segundo paso
	\item Tercer paso
\end{enumerate}


Se puede indicar un tipo diferente de numeración de la siguiente forma:

%\begin{enumerate}[tipo de numero a utilizar]
\begin{enumerate}[I]
	\item Primer paso
	\item Segundo paso
	\item Tercer paso
% En esta lista incluir los pasos que se deben seguir para compilar un archivo haciendo uso del
% compilador latex
\end{enumerate}

%Numeracion romana en minusculas: \begin{enumerate}[i] 
%Numeración alfabetica en  minusculas: \begin{enumerate}[a]
%Numeracion alfabetica en mayusculas: \begin{enumerate}[A]

%\smallskip,\medskip y \bigskip producen pequeños saltos entre dos bloques de texto
\bigskip

\begin{enumerate}[a]
	\item Primer paso
	\item Segundo paso
	\item Tercer paso
\end{enumerate}

Se puede indicar el tipo de número mediante los caracteres: \textit{a, A, I, i, 1}.

Este tipo de listas también pueden anidarse, por ejemplo:

\begin{enumerate}
	\item Primer paso
		\begin{enumerate}
			\item Verificar primero ..
			\item Verfiricar después ...
				\begin{enumerate}
					\item Requisito 1
					\item Requisito 2
				\end{enumerate}
			\item Ahora ya podemos continuar.
		\end{enumerate}
	\item Segundo paso
	\item Tercer paso
\end{enumerate}

También podemos indicar el número a usar para iniciar la numeración:

\begin{enumerate}
	\setcounter{enumi}{3}
	\item Primer paso
	\item Segundo paso
	\item Tercer paso
\end{enumerate}

%\subchapter{Listas no numeradas}

Las listas no numeradas se pueden introducir mediante el entorno \textit{itemize} 

\begin{itemize}
	\item Primer ítem	
	\item Segundo ítem
	\item Tercer ítem
\end{itemize}


Podemos modificar el tipo de símbolo usado en un item de la siguiente forma:

\begin{itemize}
	\item[w] Primer ítem
	\item[$\delta$] Segundo ítem %[$\simbolo matematico$]
	\item Tercer ítem
\end{itemize}

Si deseamos cambiar el tipo de viñeta usado en los items debemos incluir las siguientes expresiones en el preámbulo.

\begin{verbatim}
\renewcommand{\labelitemi}{$-$}
\renewcommand{\labelitemii}{$\cdot$}
\end{verbatim}

De la misma forma, podemos indicar el tipo de números a usar para las listas numeradas:

\begin{verbatim}
\renewcommand{\labelenumi}{(\Roman{enumi})}
\renewcommand{\labelenumii}{\Roman{enumi}.~\alph{enumii}}
\end{verbatim}

% Nuevo capitulo
\chapter{Margenes y tamaño de la caja de texto}

Ésta sección complementa a lo capitulo \ref{primera} incluyendo ejemplos de tablas escritas en \LaTeX.

Podemos especificar el tamaño de los margenes y de la zona de texto mediante las siguientes expresiones:

\begin{verbatim}
\textheight = medida
\textwidth = medida
\topmargin = medida
\oddsidemargin= medida
\end{verbatim}

Estos tamaños deben especificarse en el preámbulo.

Se debe considerar que \LaTeX{} coloca diferente tamaño de margen para cada clase de documento.

\backmatter %Ultimas partes del documento(Bibliografia, apendices..)
% Ultimo capitulo
\chapter{Tabla de contenidos y título}

Se puede introducir una tabla de contenidos de manera automática en nuestro documento mediante la expresión:

\begin{verbatim}
\tableofcontents
\end{verbatim}

\end{document}
