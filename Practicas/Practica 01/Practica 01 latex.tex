% Aquí comienza el preámbulo
\documentclass[letterpaper,12pt]{article}

% Soporte para acentos. En Windows se puede usar latin1 o utf8, se recomienda utf8
\usepackage[utf8]{inputenc}
% Español, división de sílabas
\usepackage[spanish,mexico]{babel}
% Usar codificación T1
\usepackage[T1]{fontenc}

\usepackage{multicol}
\usepackage{enumerate}
\usepackage{hyperref}
\usepackage{setspace}

\doublespacing
%Paquetes para matemáticas
\usepackage{amsmath}		
\usepackage{amssymb} 		
\usepackage{amsfonts}
\usepackage{latexsym}

\topmargin = -2cm
\oddsidemargin= 0cm
\textheight = 23cm
\textwidth = 17cm

\title{Práctica 01 \LaTeX{}}
\author{Jesús Esteban Sánchez Alcántara}
\date{\today}

\begin{document}

\maketitle

\tableofcontents

\newpage 
\section{Hablemos un poco de \LaTeX{}}
\begin{multicols}{2}
\LaTeX{} es un sistema de composición de textos que está formado mayoritariamente por órdenes no
construidas a partir de comandos de TeX —un lenguaje «de nivel bajo», en el sentido de que sus
acciones últimas son muy elementales— pero con la ventaja añadida de «poder aumentar las
capacidades de \LaTeX{} utilizando comandos propios del TeX descritos en \textbf{\textit{The TeXbook}}
».3 4. Esto es lo que convierte a \LaTeX{} en una herramienta práctica y útil pues,a su facilidad de uso,
se une toda la potencia de TeX. Estas características hicieron que \LaTeX{} se extendiese rápidamente 
entre un amplio sector científico y técnico, hasta el punto de convertirse en uso obligado en comunicaciones
y congresos, y requerido por determinadas revistas a la hora de entregar artículos académicos.
\end{multicols}


\section{Pasos para poder inscribirse en un curso compartido por computo}
Para poder asistir a un curso impartido por computo es necesario seguir estos pasos:
\begin{enumerate}
	\item Revisar tu correo y estar atento a la convocatoria
	\item Entrar a la página \href{http://computo.fciencias.unam.mx:9090/Cursos/index}                   {\textbf{http://computo.fciencias.unam.mx:9090/Cursos/index}}
	\item Escoger el curso de tu agrado
	\item Llenar el formulario
	\item Esperar correo de respuesta \\

Si fuiste aceptado te llegara un correo de confirmación
\setcounter{enumi}{6}
\item Llega al primer día de clase y hechale ganas el resto del curso
\end{enumerate}


\section{¿Cómo obtengo mi constancia de curso?}
Para poder obtener la constancia es necesario cumplir con los siguientes requisitos:
\begin{enumerate}
	 \item[$\bigstar$] Verificar que mi nombre este bien escrito en la lista del curso, de no ser asi avisar al instructor
	 \item[$\bigstar$] Asistir al menos al 80 \% de las clases
	 \item[$\bigstar$] Entregar las practicas
	 \item[$\bigstar$] Esperar un correo de cursos.computo@ciencias.unam.mx que diga donde y a partir de que fecha puedes
	 \item[$\bigstar$] Recoger tu constancia	
	 \item[$\bigstar$] Recoger tu constancia y ser feliz :-)
\end{enumerate}


\section{Unas cuantas fórmulas matemáticas}

$$\sum_{\overset{k=0}{k\neq j}}^{\infty} a_k$$
$$z=|z|(cos \varphi + i sen \varphi) $$
$$|a cos x + b sen x| \leq \sqrt{{a}^2 + {b}^2} $$
\[\left[ \begin{array}{ccc}
	1 & \hdots & 1 \\
	\vdots & \ddots & \vdots \\
	1 & \hdots & 1 
\end{array}
\right]
\]

\[\left\{\begin{array}{c}
	x^{t^{n^{2}}}
\end{array}
\right\}
\]

$${x_n \underset{{n \to \infty}}{\longrightarrow} x}$$

\[
f(n)= \left\{ 
\begin{array}{ccc}
	n/2 & \mbox{si n es par } \\	
	3n+1 & \mbox{si n es impar } 
\end{array}
\right.
\]

\[ \lim_{x \to +\infty} f(x) \]
$$p_k(x)= \prod_{\overset{i=1}{i\neq k}}^{n} \left( \frac{x-t_i}{t_k-t_i} \right)$$

\begin{equation}
\left( \frac{1}{2} \right)= \displaystyle \left[ {1 \over 2}\right] = 
\displaystyle \left | \frac{1}{2} \right | = \left\{ \frac{1}{2} \right\} =
\left \langle \frac{1}{2} \right \rangle
\end{equation}

\[
Arreglo_7 = \left\{\begin{array}{ccccc}
	\displaystyle  \sum_{\overset{k=0}{k\neq j}}^{\infty} a_k & |z| & cos \varphi & sen \varphi & \beta \\
	n/2 & 3n+1 & f(n) & \delta & \pi \\
	\frac{1}{2} & \lim_{x \to +\infty} f(x) \ & \pi & 8 & \sqrt[5]{2 \phi^2 + \beta^3} \\	
	-9.5 & 5,8 & 17,8 & -\infty & \frac{75 x}{x-1}
\end{array}
\right\}
\]

\end{document}