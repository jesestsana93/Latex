\documentclass{article}

\usepackage[T1]{fontenc}
\usepackage[utf8]{inputenc}
\usepackage[spanish,mexico]{babel}
 
\usepackage{natbib}
\usepackage{apalike}
% Opciones aplicables a natbib
% round - Paréntesis redondos.
% square - Paréntesis cuadrados.
% curly - Llaves.
% angle - Corchetes angulados.
% colon - Separar múltiples citas con dos puntos.
% comma - Usar una coma como separador.
% authoryear - Citas authoryear.
% numbers - Citas numéricas.
% super - Citas numéricas con superscripts.
% sort - Ordenar citas múltiples en el orden en que aparecen en la lista de referencias.
% sort&compress - Igual que sort pero las citas pueden aparecer comprimidas.

%\usepackage[nottoc]{tocbibind}
 
 %Para compilar: 1. pdflatex, despues 2.bibtex y otras dos veces pdflatex
\title{Uso de bibliografía con \texttt{natbib}}
\author{Chucho Pérez}
\date{}
 
\begin{document}
 
\maketitle
 
En este documento hacemos uso del paquete \texttt{natbib} para introducir la bibliografía. 

Veremos algunos ejemplos de referencias para el libro \textit{The \LaTeX{} Companion} \citet{latexCompanion}, el artículo 
\textit{Creación de Documentos Científicos} \citep[página 90]{lamportLatex} y el sitio de \textit{CTAN} \cite{ctanWebsite}.

Las referencias \cite{latexCompanion,lamportLatex} corresponden a publicaciones en papel.

Consulte~\citet*{latexCompanion} para mayor referencia sobre tablas.

\citep[vea][página 90]{lamportLatex}, es un libro clásico de Lamport.

\citeauthor{lamportLatex}, \citeyear{lamportLatex}, \citeyearpar{lamportLatex}.

\medskip
 
%Estilos para la bibliografia(solo se usa uno)
\bibliographystyle{unsrtnat}
%\bibliographystyle{dinat}
%\bibliographystyle{humannat}
%\bibliographystyle{plainnat}
%\bibliographystyle{abbrvnat}
%\bibliographystyle{unsrtnat}
%\bibliographystyle{rusnat}
%\bibliographystyle{unsrtnat}

% Otras posibilidades son: abbrv, acm, alpha, apalike, ieeetr, plain, siam y unsrt. Éstos se pueden usar sin natbib.
%\bibliographystyle{apalike}
%\bibliographystyle{chicago}

\bibliography{articulos,libros}
 
\end{document}
