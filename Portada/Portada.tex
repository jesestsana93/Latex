\documentclass[letterpaper,12pt]{article}

\usepackage[utf8]{inputenc}% Soporte para acentos. En Windows se puede usar latin1 o utf8, se recomienda utf8
\usepackage[spanish,mexico]{babel}% Español, división de sílabas
\usepackage[T1]{fontenc}

\usepackage{graphicx}
\usepackage{amssymb}		
\usepackage{fancyhdr}
\pagestyle{fancy}
\rhead{$C 1 \mathcal{E}(\eta)C 1 \alpha \int $}

\topmargin = -2cm
\oddsidemargin= 0cm
\textheight = 23cm
\textwidth = 17cm

\begin{document}
\setlength{\unitlength}{1 cm} %Especificar unidad de trabajo
\thispagestyle{empty}
\begin{picture}(18,4)
	\put(0,0){\includegraphics[width=3cm,height=4cm]{unam.jpg}}
	\put(11.5,0){\includegraphics[width=4cm,height=4cm]{ciencias.png}}
\end{picture}
\\
\\
\begin{center}
	\textbf{{\Huge Universidad Nacional Autónoma de México}\\[2.5cm]
	{\LARGE Facultad de Ciencias}\\[3.25cm]
	{\Large Lenguajes de programación}\\[2.3cm]
	{\large Alumno:Jesús Esteban Sánchez Alcántara}}\\[2cm]
\end{center}

\newpage

\section{Problema I}


\end{document}


